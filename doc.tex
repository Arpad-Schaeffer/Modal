% il existe plusieures classes de documents
% pour des documents plus longs, vous pouvez utiliser
% book ou report
\documentclass[a4paper]{article} 



% vous pouvez changer les paramètres : voici les options dispoinibles :
% - a4paper
% - fancysections
% - notitlepage ou titlepage
% - onside ou twoside selon si vous voulez l'imprimer en recto-verso ou en recto
% - sectionmark
% - chaptermark (pour les 
% - pagenumber
% - enmanquedinspiration
% en cas de doutes, pas de doutes, la documentation est sur :
%  https://gitlab.binets.fr/typographix/polytechnique/-/blob/master/source/polytechnique.pdf
\usepackage[a4paper,  fancysections,  titlepage]{polytechnique}
\usepackage[french]{babel}
\usepackage[T1]{fontenc}
\usepackage{blindtext}
\usepackage[hidelinks]{hyperref}
\usepackage{amsmath, amssymb}


% nous avons défini deux commandes :
\newcommand{\code}[1]{%
    \mbox{\ttfamily%
        \detokenize{#1}%
    }%
}

\newcommand{\resultat}[1]{%
    \quad \rightsquigarrow \quad #1%
}

\author{Arpad Schaeffer et Malo Tamalet}
\date{\today}
\title{Rapport de Modal}
\subtitle{Physique des particules, détection de muons}%
% pour changer de logo, ajoutez l'image dans un fomat PDF
% ou image en la glissant à droite et remplacez typographix
% par le nom de l'image, si vous ne voulez pas de logo, 
% supprimze la ligne. 
\logo{typographix}




\begin{document}
	\maketitle 
	
	\tableofcontents
	
	\section{Th\'eorie et fonctionnement des instruments}

    \begin

    % Contenu de cette section : Origine des muons, scintillateurs, PMT, traitement du signal
    % Voir notes précédentes, contenu à insérer ici par blocs
    
    \subsection{Origine et propri\'et\'es des muons cosmiques}
    % Contenu détaillé sur la formation des muons, flux, spectre d\'\'energie, etc.
    
    \subsection{Principe du scintillateur organique plastique}
    % Explication du fonctionnement, fluorescence, d\'\'eclenchement, rendement lumineux
    
    \subsection{Photomultiplicateur (PMT)}
    % Description du fonctionnement interne, gain, dynodes, réponse temporelle
    
    \subsection{Traitement du signal}
    % Circuit RC, discrim., coïncidences, délai, forme d\'\'onde attendue
    
    \section{MANIPULATIONS, AVANCEMENT ET R\'ESULTATS}
    
    \subsection{Montage exp\'erimental}
    % Description de la configuration des scintillateurs, PMT, câblage, oscilloscope
    
    \subsection{Visualisation du signal et ajustement exponentiel}
    % Forme d\'\'onde + fit exponentiel
    
    \begin{figure}[H]
        \centering
        \includegraphics[width=0.6\textwidth]{chemin/vers/signal_exponentiel.pdf}
        \caption{Signal analogique typique mesur\'e sur oscilloscope et ajustement exponentiel.}
        \label{fig:signal_exp}
    \end{figure}
    
    \subsection{Mesures du taux de comptage}
    
    \subsubsection{Variation du seuil}
    \begin{figure}[H]
        \centering
        \includegraphics[width=0.7\textwidth]{chemin/vers/graph_seuil.pdf}
        \caption{Taux de comptage en fonction du seuil de discrimination.}
        \label{fig:taux_vs_seuil}
    \end{figure}
    
    \subsubsection{Variation de la haute tension}
    \begin{figure}[H]
        \centering
        \includegraphics[width=0.7\textwidth]{chemin/vers/graph_HT.pdf}
        \caption{Taux de comptage en fonction de la tension appliqu\'ee aux PMT.}
        \label{fig:taux_vs_HT}
    \end{figure}
    
    \subsection{Co\"incidences et taux de bruit}
    
    \begin{table}[H]
        \centering
        \begin{tabular}{lcc}
            \toprule
            \textbf{Configuration} & \textbf{Align\'ee} & \textbf{Avec d\'elai (fortuite)} \\
            \midrule
            Simple A & 120 /min & 118 /min \\
            Simple B & 122 /min & 121 /min \\
            Simple C & 119 /min & 120 /min \\
            A $\wedge$ B & 95 /min & 2 /min \\
            A $\wedge$ C & 93 /min & 1 /min \\
            B $\wedge$ C & 94 /min & 1 /min \\
            A $\wedge$ B $\wedge$ C & 80 /min & 0 /min \\
            \bottomrule
        \end{tabular}
        \caption{Taux de co\"incidences mesur\'es avec et sans d\'elai.}
        \label{tab:coincidences}
    \end{table}
    
    \subsection{Mesure angulaire des muons}
    
    \begin{figure}[H]
        \centering
        \includegraphics[width=0.7\textwidth]{chemin/vers/hist_angle.pdf}
        \caption{Histogramme des angles d\'incidence des muons.}
        \label{fig:hist_angle}
    \end{figure}
    
    \section{RETOUR CRITIQUE}
    
    % Discussion sur la qualité des mesures, la reproductibilité, les limites instrumentales, extensions possibles, etc.
    
    \begin{thebibliography}{}
        \bibitem{knoll} Glenn F. Knoll. \textit{Radiation Detection and Measurement}, 4e \'edition.
    \end{thebibliography}
	
\end{document}